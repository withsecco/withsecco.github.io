\documentclass[10pt, letterpaper]{article}

% Packages:
\usepackage[
    ignoreheadfoot, % set margins without considering header and footer
    top=2 cm, % seperation between body and page edge from the top
    bottom=2 cm, % seperation between body and page edge from the bottom
    left=2 cm, % seperation between body and page edge from the left
    right=2 cm, % seperation between body and page edge from the right
    footskip=1.0 cm, % seperation between body and footer
    % showframe % for debugging 
]{geometry} % for adjusting page geometry
\usepackage{titlesec} % for customizing section titles
\usepackage{tabularx} % for making tables with fixed width columns
\usepackage{array} % tabularx requires this
\usepackage[dvipsnames]{xcolor} % for coloring text
\definecolor{primaryColor}{RGB}{0, 0, 0} % define primary color
\usepackage{enumitem} % for customizing lists
\usepackage{fontawesome5} % for using icons
\usepackage{amsmath} % for math
\usepackage[
    pdftitle={QI_Yu_CV},
    pdfauthor={QI Yu},
    pdfcreator={LaTeX with RenderCV},
    colorlinks=true,
    urlcolor=primaryColor
]{hyperref} % for links, metadata and bookmarks
\usepackage[pscoord]{eso-pic} % for floating text on the page
\usepackage{calc} % for calculating lengths
\usepackage{bookmark} % for bookmarks
\usepackage{lastpage} % for getting the total number of pages
\usepackage{changepage} % for one column entries (adjustwidth environment)
\usepackage{paracol} % for two and three column entries
\usepackage{ifthen} % for conditional statements
\usepackage{needspace} % for avoiding page brake right after the section title
\usepackage{iftex} % check if engine is pdflatex, xetex or luatex

% Ensure that generate pdf is machine readable/ATS parsable:
\ifPDFTeX
    \input{glyphtounicode}
    \pdfgentounicode=1
    \usepackage[T1]{fontenc}
    \usepackage[utf8]{inputenc}
    \usepackage{lmodern}
\fi

\usepackage{charter}

\usepackage{xeCJK} % 中文支持

\setCJKmainfont{Songti SC} 

% Some settings:
\raggedright
\AtBeginEnvironment{adjustwidth}{\partopsep0pt} % remove space before adjustwidth environment
\pagestyle{empty} % no header or footer
\setcounter{secnumdepth}{0} % no section numbering
\setlength{\parindent}{0pt} % no indentation
\setlength{\topskip}{0pt} % no top skip
\setlength{\columnsep}{0.15cm} % set column seperation
\pagenumbering{gobble} % no page numbering

\titleformat{\section}{\needspace{4\baselineskip}\bfseries\large}{}{0pt}{}[\vspace{1pt}\titlerule]

\titlespacing{\section}{
    % left space:
    -1pt
}{
    % top space:
    0.3 cm
}{
    % bottom space:
    0.2 cm
} % section title spacing

\renewcommand\labelitemi{$\vcenter{\hbox{\small$\bullet$}}$} % custom bullet points
\newenvironment{highlights}{
    \begin{itemize}[
        topsep=0.10 cm,
        parsep=0.10 cm,
        partopsep=0pt,
        itemsep=0pt,
        leftmargin=0 cm + 10pt
    ]
}{
    \end{itemize}
} % new environment for highlights


\newenvironment{highlightsforbulletentries}{
    \begin{itemize}[
        topsep=0.10 cm,
        parsep=0.10 cm,
        partopsep=0pt,
        itemsep=0pt,
        leftmargin=10pt
    ]
}{
    \end{itemize}
} % new environment for highlights for bullet entries

\newenvironment{onecolentry}{
    \begin{adjustwidth}{
        0 cm + 0.00001 cm
    }{
        0 cm + 0.00001 cm
    }
}{
    \end{adjustwidth}
} % new environment for one column entries

\newenvironment{twocolentry}[2][]{
    \onecolentry
    \def\secondColumn{#2}
    \setcolumnwidth{\fill, 4.5 cm}
    \begin{paracol}{2}
}{
    \switchcolumn \raggedleft \secondColumn
    \end{paracol}
    \endonecolentry
} % new environment for two column entries

\newenvironment{threecolentry}[3][]{
    \onecolentry
    \def\thirdColumn{#3}
    \setcolumnwidth{, \fill, 4.5 cm}
    \begin{paracol}{3}
    {\raggedright #2} \switchcolumn
}{
    \switchcolumn \raggedleft \thirdColumn
    \end{paracol}
    \endonecolentry
} % new environment for three column entries

\newenvironment{header}{
    \setlength{\topsep}{0pt}\par\kern\topsep\centering\linespread{1.5}
}{
    \par\kern\topsep
} % new environment for the header

\newcommand{\placelastupdatedtext}{% \placetextbox{<horizontal pos>}{<vertical pos>}{<stuff>}
  \AddToShipoutPictureFG*{% Add <stuff> to current page foreground
    \put(
        \LenToUnit{\paperwidth-2 cm-0 cm+0.05cm},
        \LenToUnit{\paperheight-1.0 cm}
    ){\vtop{{\null}\makebox[0pt][c]{
        \small\color{gray}\textit{Last updated in September 2024}\hspace{\widthof{Last updated in September 2024}}
    }}}%
  }%
}%

% save the original href command in a new command:
\let\hrefWithoutArrow\href

% new command for external links:


\begin{document}
    \newcommand{\AND}{\unskip
        \cleaders\copy\ANDbox\hskip\wd\ANDbox
        \ignorespaces
    }
    \newsavebox\ANDbox
    \sbox\ANDbox{$|$}

    \begin{header}
        \fontsize{25 pt}{25 pt}\selectfont QI Yu

        \vspace{5 pt}

        \normalsize
        \mbox{Hangzhou, Zhejiang, China}%
        \kern 5.0 pt%
        \AND%
        \kern 5.0 pt%
        \mbox{\hrefWithoutArrow{mailto:3210101154@zju.edu.cn}{3210101154@zju.edu.cn},\hrefWithoutArrow{mailto:withsecco@gmail.com}{withsecco@gmail,com} }%
        \kern 5.0 pt%
        \AND%
        \kern 5.0 pt%
        \mbox{\hrefWithoutArrow{tel:+86 19550212355}{+86 19550212355}}%
        \kern 5.0 pt%
        \\%
        \kern 5.0 pt%
        \mbox{\hrefWithoutArrow{https://withsecco.github.io/}{Homepage}}%
        \kern 5.0 pt%
        \AND%
        \kern 5.0 pt%
        \mbox{\hrefWithoutArrow{https://www.linkedin.com/in/yu-qi-938499277}{Linkedin}}%
        \kern 5.0 pt%
        \AND%
        \kern 5.0 pt%
        \mbox{\hrefWithoutArrow{https://github.com/withsecco}{Github}}%
    \end{header}

    \vspace{5 pt - 0.3 cm}

% \section{Welcome to RenderCV!}


        
%     \begin{onecolentry}
%         \href{https://rendercv.com}{RenderCV} is a LaTeX-based CV/resume version-control and maintenance app. It allows you to create a high-quality CV or resume as a PDF file from a YAML file, with \textbf{Markdown syntax support} and \textbf{complete control over the LaTeX code}.
%     \end{onecolentry}

%     \vspace{0.2 cm}

%     \begin{onecolentry}
%         The boilerplate content was inspired by \href{https://github.com/dnl-blkv/mcdowell-cv}{Gayle McDowell}.
%     \end{onecolentry}



% \section{Quick Guide}

% \begin{onecolentry}
%     \begin{highlightsforbulletentries}

%         \item Each section title is arbitrary and each section contains a list of entries.

%         \item There are 7 unique entry types: \textit{BulletEntry}, \textit{TextEntry}, \textit{EducationEntry}, \textit{ExperienceEntry}, \textit{NormalEntry}, \textit{PublicationEntry}, and \textit{OneLineEntry}.

%         \item Select a section title, pick an entry type, and start writing your section!

%         \item \href{https://docs.rendercv.com/user_guide/}{Here}, you can find a comprehensive user guide for RenderCV.


%         \end{highlightsforbulletentries}
%     \end{onecolentry}


\section{Education}


    
    \begin{twocolentry}{
        Sept 2021–June 2025
    }
        \textbf{Zhejiang University}, B.A., English(Linguistics)\end{twocolentry}

    \vspace{0.10 cm}
    \begin{onecolentry}
        \begin{highlights}
            \item GPA: 3.94/4.00% (\href{https://example.com}{a link to somewhere})
            \item Coursework: Psycholinguistics (99), Introduction to Cognitive Neuroscience (95), Signals and Systems in Psychology (94), Elements of Bayesian Statistics (93), Modern Linguistics (93)
        \end{highlights}
    \end{onecolentry}

    \vspace{0.3 cm}

    \begin{twocolentry}{
       2025 Fall
    }
        \textbf{Zhejiang University}, M.A., Linguistics\end{twocolentry}
    
    \vspace{0.10 cm}
    \begin{onecolentry}
        \begin{highlights}
            \item Concentrations: Neurolinguistics, Psycholinguistics
            \item Supervisor: Prof.YANG Jing
        \end{highlights}
    \end{onecolentry}

\section{Research Experience}


\textbf{Research Center for Life Science Computing}, Zhejiang Lab

\vspace{0.10 cm}

    \begin{twocolentry}{
        April 2024-present
    }
        \textit{Research Assistant, Project-based Staff}
        \item Supervisor: Dr.LUO Cheng
    \end{twocolentry}

    \vspace{0.10 cm}
    \begin{onecolentry}
        \begin{highlights}
            \item Currently working on a project analyzing auditory cognitive mechanisms and evaluating the brain-like performance of an auditory model
            \item Contributing to multimodal dataset development, including EEG, EOG, fMRI data acquisition
            \item 4th author of patent “Professional Domain Database Retrieval Method, Electronic Equipment and Media based on Knowledge Enhancement of Large Language Model” [基于大语言模型知识增强的专业领域数据库检索方法、电子设备、介质] (under review)
            \item \textbf{Tools Used:} MATLAB, Psychtoolbox, EEG, fMRI
        \end{highlights}
    \end{onecolentry}


    \vspace{0.3 cm}

    \textbf{Covid-19 English Neologisms Database Consturction and Word-Formation and Semantics Studies[基于语料库的新冠英语新词数据库建设及其构词和语义研究]}, Student Research Training Program (SRTP)

    \vspace{0.10 cm}

    \begin{twocolentry}{
        April 2023–April 2024
    }
       \textit{Research Partner (2-person team)}
        \item Supervisor: Prof.SHAO Bin
    
    \end{twocolentry}

    \vspace{0.10 cm}
    \begin{onecolentry}
        \begin{highlights}
            \item \textbf{National level SRTP, granted highest level of SRTP funding}
            \item Collected 1,400 news articles related to COVID-19 from January 1, 2021, to June 30, 2023 and Used the NLTK and spaCy toolkits to process the texts and extract new words
            \item Created a neologism database that included information on usage frequency, domain of use, word formation process, definitions, examples, etc.
            \item Applied cognitive linguistics theories, such as Conceptual Integration Theory, to explore how the metaphorical and metonymic properties and polysemy of vocabulary reflect public perceptions and responses during the COVID-19 pandemic
            \item \textbf{Tools Used:} Python, SpaCy, NLTK
        \end{highlights}
    \end{onecolentry}
    
\section{Course Projects}

    \begin{samepage}
        
        \textbf{Does Interactional Context during Comprehension Modulate Bilingual’s Cognitive Control?}

        \vspace{0.10cm}

        \begin{twocolentry}{
            June 2024
        }
            \textit{Psycholinguistics}
        \end{twocolentry}

         \vspace{0.10 cm}

         \begin{highlights}
            \item Adopted a dual-task paradigm proposed by Adaptive Control Hypothesis (ACH) that included a language comprehension trial to create the three interactional contexts (single language, dual language and dense code-switching) and a non-linguistic task (i.e., a non-verbal Flanker task)   
            \item Coded the experiment using Psychopy and Collected behavioral data (RT and Acc) from 28 participants; Used repeated-measures ANOVA to analyze the Flanker Effect and overall RT across the three contexts   
            \item \textbf{Tools Used:} Psychopy, SPSS
         \end{highlights}
    \end{samepage}
    
    \newpage
      
    \begin{samepage}
        \textbf{Functional Connectivity of resting state Language Networks in Patients with Autism Spectrum Disorders}

        \vspace{0.10cm}

        \begin{twocolentry}{
            June 2024
        }
            \textit{Introduction to Cognitive Neuroscience}
        \end{twocolentry}

        \vspace{0.10 cm}

        \begin{highlights}
            \item Used ABIDE-II datasets and pre-processed the rs-fMRI data of 48 participants (22 TCs and 26 ASDs); Selected ROIs and conducted seed-based functional connectivity analysis with CONN  
            \item Results showed significant differences in the left pSTG functional connectivity between the ASD group and the control group (p-FDR< 0.05), ASD group exhibited atypical brain lateralization in the ASD group, supporting previous research findings  
            \item \textbf{Tools Used:} MATLAB, SPM, CONN
        \end{highlights}
    \end{samepage}


\section{Awards \& Certifications }

        \vspace{0.2 cm}
        \begin{twocolentry}{
            2024
        }
            \textbf{Zhejiang University Third Prize Scholarship}\end{twocolentry}

        \vspace{0.1 cm}

        \begin{twocolentry}{
            2024
        }
            \textbf{National Level Student Research Training Program (SRTP)}\end{twocolentry}
            Granted the highest level of SRTP funding and recognized for outstanding project completion

        \vspace{0.1 cm}

        \begin{twocolentry}{
            2021
        }
            \textbf{Zhejiang University Global Engagement Program(GEP)}\end{twocolentry}
            Honor Class of 2025
        
\section{Skills}
        
        \begin{onecolentry}
            \item \textbf{Progamming:} Python, R, MATLAB, LaTeX
            \item \textbf{Packages:} Psychopy, Psychtoolbox, SPM, CONN, JAGS (bayesian modelling), SpaCy, NLTK
            \item \textbf{Methods:} behavioral; EEG(conducting and data anallysis);fMRI(conducting and data analysis)
            \item  \textbf{Languages:} Chinese(native), English(fluent), German(beginner)

        \end{onecolentry}

\end{document}

